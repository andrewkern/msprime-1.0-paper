% LaTeX rebuttal letter example.
%
% Copyright 2019 Friedemann Zenke, fzenke.net
%
% Based on examples by Dirk Eddelbuettel, Fran and others from
% https://tex.stackexchange.com/questions/2317/latex-style-or-macro-for-detailed-response-to-referee-report
%
% Licensed under cc by-sa 3.0 with attribution required.

\documentclass[11pt]{article}
\usepackage[utf8]{inputenc}
\usepackage{lipsum} % to generate some filler text
\usepackage{fullpage}

% import Eq and Section references from the main manuscript where needed
% \usepackage{xr}
% \externaldocument{manuscript}

% package needed for optional arguments
\usepackage{xifthen}
% define counters for reviewers and their points
\newcounter{reviewer}
\setcounter{reviewer}{0}
\newcounter{point}[reviewer]
\setcounter{point}{0}

% This refines the format of how the reviewer/point reference will appear.
\renewcommand{\thepoint}{\thereviewer.\arabic{point}}

% command declarations for reviewer points and our responses
\newcommand{\reviewersection}{\stepcounter{reviewer} \bigskip \hrule
                  \section*{Reviewer \thereviewer}}

\newenvironment{point}
   {\refstepcounter{point} \bigskip \noindent {\textbf{Reviewer~Point~\thepoint} } ---\ }
   {\par }

\newcommand{\shortpoint}[1]{\refstepcounter{point}  \bigskip \noindent
    {\textbf{Reviewer~Point~\thepoint} } ---~#1\par }

\newenvironment{reply}
   {\medskip \noindent \begin{sf}\textbf{Reply}:\  }
   {\medskip \end{sf}}

\newcommand{\shortreply}[2][]{\medskip \noindent \begin{sf}\textbf{Reply}:\  #2
    \ifthenelse{\equal{#1}{}}{}{ \hfill \footnotesize (#1)}%
    \medskip \end{sf}}

\begin{document}

\section*{Response to the editor}
% General intro text goes here
Thank you for considering this manuscript for publication. We are glad that
Reviewers 1 and 2 are enthusiastic about this community effort, and their reviews
have helped to resolve a number of issues with the manuscript. We are
disappointed that although Reviewer 3 agrees in the value of the software,
they do not believe the manuscript
is appropriate for publication in Genetics. However, as we point out in the
response, it is unclear whether they were reviewing according to the
``Methods, technology, and resources`` section scope.

\subsection*{Specific comments}
% We address your specific comments below:
% % \reviewersection

\textit{
It is most important that you address the following in your resubmission:
Improve the structure and focus of the main body of the manuscript. This should
include moving a significant portion of the material to appendices or
supplemental material, and adding an up-front summary of the main new features
of msprime (relative to the 2016 paper).}

We have shortened the manuscript by moving the details of the analysis
of Hudson's algorithm and selective sweeps to appendices, and deleting
unnecessary text in several sections. We have added Table 1, which
summarises the features added to msprime since the 2016 paper.

\section*{Response to the reviewers}
% General intro text goes here
We thank the reviewers for their close reading of our manuscript and
insightful comments. In the following we address the points raised
in turn.

% Let's start point-by-point with Reviewer 1
\reviewersection

% Point one description
\begin{point}
Simulation is a vital part of population genetics, and msprime has quickly
become a crucial tool for many researchers. msprime was first presented in a
publication in 2016, where the main focus was the use of "sparse trees" and
"coalescence records" to greatly increase the efficiency and practicality of
large coalescent simulations. In this new manuscript, the authors introduce a
new version of the software, with this manuscript ""focusing on the aspects
that are either new for this version, or in which our approach differs
significantly from classical methods."

Since 2016, msprime has gained popularity not only due to its computational
efficiency, but also for its software development style with well-documented
features, development guidelines, and ever-growing capabilities. The extensive
author list shows how the community around msprime and associated programs has
grown over the last five years. The current manuscript has a huge number of
citations (I count ~178!), generally representing a willingness to engage with
a broad spectrum of the adjacent scientific community.

The manuscript covers many features of the 1.0 version of msprime including at
least (my list and organization):

Software / API / innovations:
Clear separations of ancestry and mutation simulations
Improved methods for downstream analyses
Integration with other software including other simulators
Estimating run times of sim\_ancestry
ARG recording
Specific API for specifying demographic histories
Software development model

Implementations of external methods:
Selective sweeps
Instantaneous bottlenecks
Multiple merger coalescents
Discrete time Wright-Fisher
Gene conversion
Mutation models

As well as benchmarks against other software implementations.

The manuscript is well written and comprehensive - I do not find any major
deficiencies. This manuscript handles a wide range of topics with a high degree
of clarity. And I find the manuscript certainly reaches the criteria "... tools
of interest to a wide range of geneticists" and so is a good fit as an
investigation within the "Methods, Technology, \& Resources" subject.
\end{point}

\begin{reply}
Thank you for the comprehensive summary. We are delighted that these community
efforts are so well received.
\end{reply}


\begin{point}
However, I do think it would be possible to increase the clarity and usefulness
of the manuscript with some relatively small changes in organization.
\end{point}

\begin{reply}
Thank you for the detailed review and suggestions, which we hope have resulted
in an improved manuscript.
\end{reply}

\begin{point}
As, is, the manuscript lacks a simple overview of the new capabilities of
version 1.0. Instead these are described in separate sections. The manuscript
might benefit from a short summary and or table of the software's basic
capabilities, API innovations, new functionality, etc. This may also provide a
way to structure the larger flow of the paper, which currently seems to touch
on the different topics in a somewhat random order.
\end{point}

\begin{reply}
We have added table 1 to summarise the new features.
\end{reply}

\begin{point}
Generally, I found the figures and examples appropriate and well done. However,
I think they could benefit from more cohesive presentations and descriptions. I
will provide a number of examples - figure 3 and figure 4 present running times
of various aspects of msprime, but present similar data in different ways. In
Figure 3, the data points are not presented directly, while they are present on
Figure 4. And based on the descriptions sometimes replicates are summed over
(Figure 7) or averaged over (Figure 8?) to determine the values on the y-axis.
Figure 3 clearly specifies sim\_mutations as the input for the benchmark, while
for most other figures (e.g. 4) it frequently just says something like "running
time for msprime." Sometimes only population-size scaled rates are provided in
descriptions (e.g Figure 5 and 7) while in others both scaled and absolute
sizes are provided. Sometimes "samples" are described to be diploid (figure 1)
or sometime haploid (Figure 3), but other times the ploidy is left implied (e.g
Figure 4). Sometimes the population size is referred to "effective population
size" (e.g Figure 4) and other times just "population size" (Figure 3). More
consistency in these simulations, descriptions, and figures could add clarity
to the presentation.
\end{point}

\begin{reply}
Thank you for examining the figures in this detail and providing such
helpful feedback. We have tried to improve the presentation by taking
the following steps for the main text figures:

\begin{itemize}

\item All plots now contain data markers to help show where the data
points are and to distinguish the lines from each other.

\item All plots state they are running ``sim\_ancestry`` or ``sim\_mutations``.

\item Stated explicitly whether samples are haploid or diploid in the
figure captions, and updated axis labels also.

\item All figures now show the time averaged across replicates.

\end{itemize}

In addition figure 4 has been moved to the appendix, making the remaining
figures much more uniform in appearance and purpose (it is a special
case because it is showing how well theoretical predictions map
to observations, as well as those observations).
\end{reply}

\begin{point}
I did find the lack of a disk storage benchmark an
interesting choice, as this aspect of the software is highlighted in the
introduction but is not present in the manuscript.
\end{point}

\begin{reply}
We did not include any benchmarks of storage space as this was explored
in detail in the 2019 Nature Genetics paper, and we directly reference
this paper when discussing the storage efficiency of tree sequences.
\end{reply}

\begin{point} Figure comments
Figure 2
I think it could be made more clear that the genotype matrix is not part of the set of tables.
\end{point}

\begin{reply}
We agree, and have revised Figure 2 and the caption to clarify the distinction between
stored and derived data.
\end{reply}

\begin{point}
Figure 4
It is not clear if these two plots present all the data used for the quadratic
fits, or if these data continue outside of the plots. Samples are described,
but not their ploidy, are they haploids or diploid? From the plots here, it is
not clear if the quadratic relationship holds for sample sizes $\ll$ 1000. I
suggest this figure should also include a smaller sample size, (eg n=10)
\end{point}

\begin{reply}
We have moved Figure 4 to the appendix, as it is of more specialised interest
than the other main text figures. We have truncated the x-axis on the
quadratic fits to just the available data, to clarify that we are showing
all of the data. We have clarified that samples are diploid.

The plot is already quite complex and it is difficult to
see how we could add a third line without major changes. The figure is
primarily intended as a useful yardstick for users who wish to run simulations
and would like a rough idea of how long it should take, and therefore adding
a third line for very small sample sizes is likely to make it less useful
for this purpose.
\end{reply}

\begin{point}
Figure 5
I suggest including the absolute Ne and gc rates be included in the legend, in
addition to the scaled rate.
\end{point}

\begin{reply}
We have redone this figure with the exact estimates from Lapierre et al
and quoted the absolute Ne and gc values.
\end{reply}

\begin{point}
Figure 7
Why is an order of magnitude lower recombination rate used here than in other
examples? This choice seems arbitrary and makes it difficult to relate these
benchmarks results to any others in the paper. Samples are diploid or haploid?
I would suggest the absolute Ne and s rates be included in the legend, in
addition to the scaled rate.
\end{point}

\begin{reply}
We have redone this figure to use the same recombination rate and population
sizes as the rest of the paper, clarified that samples are diploid, given
results in terms of the average per replicate and generally made the plot
as close to the others as possible.
\end{reply}

\begin{point}
Figure 8
Averaged points are referenced but not shown "Each point...". The plot could be
more clear where the data points are and where the lines are interpolating
between data points. Because of how the simulations were described and plotted,
I found it awkward to try to compare the running time of the DTWF and
coalescent models, even if this was not the purpose of the current figure. The
figure says "to ensure we are measuring" I would suggest changing the text to
"with a goal of measuring".
\end{point}

\begin{reply}
We have update this figure (and all others in the main text) to include
data markers. We have also rephrased the caption to try to clarify the
parameters simulated and make them consistent with other figures.
\end{reply}

\begin{point}
Individual line comments
Line 142 - what is the "msp" program? Is this a command line version?
\end{point}

\begin{reply}
We have clarified this point with a parenthetical comment.
\end{reply}

\begin{point}
Line 370 - the text here makes it sound like Kelleher et al (2016) makes a
statement relating eq. 1, but the Hein et al (2004) is not cited in the Keleher
et al (2016).
\end{point}

\begin{reply}
We have rephrased this to say "(see also Kelleher et al., 2016, Fig. 2)".
\end{reply}

\begin{point}
Line 370 - population size(s)
\end{point}

\begin{reply}
Thank you, we have corrected this sentence in the revised manuscript.
\end{reply}

\begin{point}
Line 420 - Despite the long description, I was not able to follow how the
previous paragraph implies that "work is spread out relatively evenly on the
coalescent time scale"
\end{point}

\begin{reply}
This is implied by the fact that the number of lineages decreases slowly
on the coalescent timescale. However, the point is not crucial to the
main goal of the paragraph, and we have therefore deleted the sentence
for clarity.
\end{reply}

\begin{point}
Line 422 - There is no context to what is meant by "large" or "small" here - is
2 or 10 or 10000 "large".
\end{point}

\begin{reply}
TODO
\end{reply}

\begin{point}
Line 607 - The equations for the allele frequencies during sweeps seem slightly
out of place as (I assume) these are not novel to msprime. There are many
places where more details could be provided about methods discussed here, but
in other cases the equations are left for the supplement or other papers.
\end{point}

\begin{reply}
We have moved the sweep trajectory equations to an appendix.
\end{reply}

\reviewersection

\begin{point}
Kelleher and colleagues present the 1.0 version of 'msprime', the leading
coalescent simulator which is built upon a highly memory- and time-efficient
data structure, introduced by the lead author, which supports exact simulation
under the coalescent model with recombination.

In addition to being highly efficient, the software engineering of msprime is
of the highest quality. The program includes extensive unit and validation
tests, and offers an API that simplifies worksflows and reduces the
inefficiency and sources of error associated with large intermediate text
files.

Kelleher has made every effort to engage with the community, with the result
that msprime now support many specialized features that have been contributed
by other developers. This has the important benefit of reducing the chances
that this software will lack support in future.

In short, this is an admirable project, and a shining example of successful
academic software development.
\end{point}

\begin{reply}
Thank you for the kind words, we are delighted that these community efforts are so well
received.
\end{reply}

\begin{point}
The paper itself however is less polished. It is very long (37 pages / 1200
lines without appendix), mostly because it is in places overly detailed, and
seems to err on the side of completeness rather than conciseness or clarity.
\end{point}

\begin{reply}
Thank you for this well-founded criticism. We have reduced the length of the manuscript
by moving some material to appendices, cutting ``documentation-like'' text,
and tried to address specific points raised below. We hope that this has
resulted in an improved manuscript.
\end{reply}

\begin{point}
For instance, lines 160-177 describe the data structure in some detail, while
this is published information. Line 192 remarks that "storage space is
dramatically reduced", and goes on to make this more precise and point out some
(fairly obvious) advantages in lines 193-202.
\end{point}

\begin{reply}
It is true that the details of the succinct tree sequence have been laid out
in earlier publications, and it is reassuring that the advantages are
obvious to you. However, for many of the intended audience these methods will
be new, and we feel it's important that this paper is a self-contained,
thorough, and convincing argument for the advantages of tree sequences and the
tskit library in simulation workflows.
Very many papers are still published using ms and other
simulators that use inefficient data formats, and it is these users that
we want to convince. We also hope to convince those that might be developing
their own simulators to take advantage of tskit and its extensive
functionality, rather then developing everything from scratch, as is the
classical and still dominant approach.
\end{reply}

\begin{point}
As another example, line 301-4
states that "Simulating mutations ... is efficient" and refers to Fig 3 for
evidence. The rest of the section (304-319) details a number of examples shown
in Fig 3 that do not add much to the story.
\end{point}

\begin{reply}
We have removed the some of the examples and shortened the section.
\end{reply}

\begin{point}
As a third example, in the
recombination section after useful comparisons with other approaches, line 354
states that the proposed algorithm is still quadratic in the recombination
rate. The long section from line 354 to 426 details the algorithmic reasons,
and use various analytic approximations and simulations to support this. This
material is of interest only to a few algorithm developers, and I would imagine
it would be better placed in an appendix, with the key observations summarized
in the main text.
\end{point}

\begin{reply}
We have moved this detailed discussion to an appendix.
\end{reply}

\begin{point}
Another feature of the current paper is that a number of topics return several
times with minor variations. This is particularly true for algorithmic
efficiency, and the API, both of which are discussed multiple times. It is of
course true that algorithmic efficiency is the major distinguishing feature of
msprime; nevertheless, the focus of the current paper seems to be the rich
feature set, and the software development model, as well as the various APIs
that allow for a more integrated and less error prone workflow. I would suggest
to introduce separate sections for algorithmic efficiency and the API, and
discuss the various features (as you do now) in their own separate section, but
focus on these features from a user's rather than a developer's perspective.
\end{point}

\begin{reply}
The large number and diversity of features in msprime has made the
organisation of this
manuscript a major challenge, and the present layout of a series of
relatively self-contained topics arrived at after much experimentation.
Issues like interface design and algorithmic efficiency necessarily
cut across most of these topics. Separate sections on efficiency
and API design would need to refer to these other sections for context,
and would lead, we believe, to a longer and less cohesive document.

However, we do agree that some sections are too long and this criticism
has been very helpful to allow us see where content could be cut
without loss of information.
For example, we have cut out references to the API design in the
Simulating Mutations section, and substantially cut back the Demography
section (see next point).
\end{reply}

\begin{point}
Occasionally the paper reads more like documentation than a paper, such as when
the Demography class is introduced and several associated functions are
mentioned - this is not relevant for a paper in my opinion.
\end{point}
\begin{reply}
This is a very helpful observation, and helped us clarify where
text can be cut. We have substantially cut back the Demography section
and other areas where we were veering towards documentation.
\end{reply}

\begin{point}
The section about the development model is interesting - very few papers
include such a section, and indeed I think this is one of the stand-out
features of this project that it is so successfully supported by many
contributors. My only suggestion here is to remove the listing of number of
lines of code - this is a very questionable proxy for quality or quantity of
work, and the work is impressive as it is.

\end{point}
\begin{reply}
Thank you for your suggestion, we have removed the number of lines of code
in the revised manuscript.
\end{reply}

\reviewersection

\begin{point}
This manuscript introduces a simulation tool, msprime version 1.0, and
describes features, additions, and performance of the simulator relative to the
earlier msprime software. I will begin by saying that I believe all researchers
in population genomics who use simulations have appreciated the quality work
that has been previously done in the msprime sphere. The contributions proposed
here clearly improve on a number of features, and that of course has value.
\end{point}
\begin{reply}
Thank you, we appreciate the kind words.
\end{reply}

\begin{point}
However, I did not find this work to be appropriate for Genetics, and would
rather view it as more appropriate for a journal like Bioinformatics. The
fundamental reason is that this basically appears to perform old
functionalities faster, and the added functionalities are modest. Again, both
are valuable, but this work doesn't address much larger biological /
statistical inference problems that are in need of development in this area,
which could justify publication in a journal like Genetics.
\end{point}
\begin{reply}
We respectfully disagree, and believe that this paper is both of substantial
interest to the readership of Genetics, and presents a number of significant
innovations that will enable years of research to follow.
These innovations either
allow simulation of models for which there is no other available simulator
(for example, the $\Lambda$-coalescent models and instantaneous bottlenecks)
or at a genuinely transformative performance differential from existing
methods. For example, msprime can simulate E-coli genomes tens of
times faster than specialised bacterial simulators, and selective
sweeps multiple orders of magnitude faster than discoal.
These are not minor incremental improvements, and fundamentally shift
the realms of possibility for population genetic simulations.
\end{reply}

\begin{point}
Specifically, the need of the field is really to simulate more realistic models
in a coalescent framework, rather than the slower forward-in-time choice that
stands as the only option presently. For example, the practical use of a tool
that only simulates strictly neutral histories, or histories with the addition
of positive selection, is likely rather limited empirically. By which I mean,
the functional regions of a genome of course experience purifying selection ,
but there is growing appreciation of the fact that this may result in
background selection effects that may be widespread across the genomes of many
species. In order for the strictly neutral demographic simulations to be of
value, one must identify genomic regions that are not only neutral but are also
unaffected by selection at linked sites. In a great many organisms, such
regions may well not exist at all; in terms of commonly studied species, these
regions may only exist in a handful of large, coding-sparse vertebrate and
plant genomes (though, even if these regions exist, they are not necessarily
easy to identify, or sufficiently large to perform neutral demographic
estimation). I note that the manuscript avoids any mention or discussion of
this problem, and avoids any citations that quantify this issue, but the
neglect doesn't alleviate the problem. The authors should really consider,
discuss, and utilize the following work carefully in this regard:

Lohmueller et al. 2011. Natural selection affects multiple aspects of genetic
variation at putatively neutral sites across the human genome. PLOS Genetics.

Comeron. 2014. Background selection as baseline for nucleotide variation across
the Drosophila genome. PLOS Genetics.

Elyashiv et al. 2016. A genomic map of the effects of linked selection in
Drosophila. PLOS Genetics.

Comeron. 2017. Background selection as a null hypothesis in population
genomics: insights and challenges from Drosophila studies. PTRSB.

Pouyet et al. 2018. Background selection and biased gene conversion affect more
than 95\% of the human genome and bias demographic inference. eLife.

Torres et al. 2018. Human demographic history has amplified the effects of
background selection across the genome. PLOS Genetics.

Campos \& Charlesworth. 2019. The effects on neutral variability of recurrent
selective sweeps and background selection. Genetics.

Jensen et al. 2019. The importance of the neutral theory in 1968 and 50 years
on: a response to Kern and Hahn 2018. Evolution.

Castellano et al. 2020. Impact of mutation rate and selection at linked sites
on DNA variation across the genomes of humans and other Homininae. GBE.

Torres et al. 2020. The temporal dynamics of background selection in
nonequilibrium populations. Genetics.

Johri et al. 2021. The impact of purifying and background selection on the
inference of population history: problems and prospects. MBE.

Murphy et al. 2021. Broad-scale variation in human genetic diversity levels is
predicted by purifying selection on coding and non-coding elements. bioRxiv.
\end{point}

\begin{reply}
It is, of course, absolutely correct to point out that neutral simulations
are an approximation, and that this approximation will be highly inaccurate
and inappropriate in many situations. However, it is reasonable to say that
it is both a widely used approximation, and that the
shortcomings of the neutral approximation have been thoroughly discussed
elsewhere. It is surely beyond the scope of this paper
to settle a decades long debate.
\end{reply}

\begin{point}
Relatedly, for the simulations of positive selection, while it has certainly
been common practice in the field for a few decades, it is rather odd to model
a single positively selected site on a background of all other neutral sites.
Positive selection may act on mutations in functional regions, and functional
regions will be largely shaped by purifying selection. Furthermore, the
Hill-Robertson effects between the modeled positively selected mutation and the
neglected array of negatively selected mutations is often non-trivial. In that
sense, the positive selection model is probably only of empirical relevance in
organisms with something approaching free recombination (to eliminate such
linkage effects), as in perhaps HIV or the like. But of course, these sorts of
organisms tend to have small, coding-dense genomes and experience very strong
selective pressures, thus returning to the first problem mentioned above that
regions free of background selection effects may not exist in the first place.
\end{point}
\begin{reply}
This is a fair point and absolutely correct. However, we have simply
implemented the standard approach that has been used and studied extensively
for decades, and have not made any claims beyond this.
\end{reply}

\begin{point}
In summary, I believe this to be a nice software addition that belongs as a
Note in a bioinformatics journal. The software solution that would represent an
important biological advance for Genetics would involve simulating more
realistic models as discussed above (as forward simulators are currently
capable of doing), but at a fast enough rate to be applicable to large-scale
simulation-based genomic inference.
\end{point}

\begin{reply}
Thank you for taking the time to explain your perspective. While we disagree on
the value of this manuscript and its potential interest to the Genetics
readership, we do agree that there is a large gap in present-day simulation
methods and that devising more sophisticated approaches to selection in
particular is a vital avenue for future work.
\end{reply}
\end{document}
